% ターミナルで $ uplatex text.tex を2回実行して,$ dvipdfmx text.dvi でPDFを生成してください.
\documentclass[dvipdfmx]{exp_report}
\titlehead{情報学群実験第4C/4i 実験レポート 第1回} % タイトルの上につける文字列.実験名などを入れる
\title{情報学群実験レポートクラスファイル利用例} % タイトル
\studentid{1250373} % 学籍番号
\name{溝口 洸熙} % 氏名
\group{Group 5C} % 所属グループ名(空白でも良い)
\institution{高知工科大学 情報学群 情報セキュリティシステム研究室} % 所属大学・研究室(空白でも良い)
\date{July 3rd, 2023} % 更新日 or 提出期限
\begin{document}
\chapter{プリアンブルの記述}
\begin{description}
    \item[\texttt{\textbackslash titlehead\{\}}] 実験タイトルの上部に記述する文字列を入力する,実験名やレポート回数など.
    \item[\texttt{\textbackslash title\{\}}] レポートのタイトルを入力する.
    \item[\texttt{\textbackslash studentid\{\}}] 学籍番号を入力する.
    \item[\texttt{\textbackslash name\{\}}] 氏名を入力する.
    \item[\texttt{\textbackslash group\{\}}] 所属グループ名を入力する.
    \item[\texttt{\textbackslash institution\{\}}] 所属大学・研究室を入力する.
    \item[\texttt{\textbackslash date\{\}}] 更新日か提出期限を入力する.
\end{description}
\chapter{相互参照}
\section{図の挿入と図の参照}
\verb|figure|環境内で\verb|\includegraphics[options]{file path}|を用いて図を挿入する.
図の場合は,図の下に\verb|\caption{}|をつける.\verb|\caption{}|の下には\verb|\label{}|を宣言し,\verb|\figref{}|相互参照する.
\begin{lstlisting}[language={tex},frame={single},basicstyle={\ttfamily}]
\begin{figure}[h]
    \centering % 中央寄せする.
    % 図の挿入,比を保ち,横幅3cmに.
    \includegraphics[keepaspectratio,width=3cm]{tiger.pdf} 
    \caption{キャプション}
    \label{fig:example}
\end{figure}
\figref{fig:example}に〜を示す.
\end{lstlisting}
\begin{figure}[h]
    \centering
    \includegraphics[keepaspectratio,width=3cm]{tiger.pdf}
    \caption{キャプション}
    \label{fig:example}
\end{figure}
\figref{fig:example}に〜を示す.
\section{表の作成と参照}
\verb|table|環境内で作成した表の参照には\verb|\tblref{}|を用いると,「表XX」と表示される.\\
例:\tblref{tbl:九州の政令指定都市}に,九州の政令指定都市を示す.
\begin{table}[h]
    \centering
    \caption{九州の政令指定都市}
    \label{tbl:九州の政令指定都市}
    \begin{tabular}{cc}
        \hline
        福岡県 & 福岡市  \\
        福岡県 & 北九州市 \\
        熊本県 & 熊本市  \\
        \hline
    \end{tabular}
\end{table}
\appendix % 付録宣言
\section{付録の付け方}
付録は,\verb|\appendix|と書く.
ソースコードを挿入するには,\verb|lstlisting|環境を使う.
\begin{lstlisting}[caption={付録です},language={python},label={src:appendix},frame={single},basicstyle={\ttfamily},numbers={left}]
print('a')
\end{lstlisting}
ソースコードの参照は,\verb|\srcref{src:appendix}|(結果:\srcref{src:appendix})のようにします.
\end{document}